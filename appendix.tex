\section{Приложения}
В таблицах записаны предположительные нижние оценки $\chi(G)$.
Через черту указаны оценка хроматического числа и число независимости
из которого эта оценка получена. Чем больше хроматическое число, тем серее
ячейка.  Описание эксперименов в начале раздела результатов.
\input{data/predict/b/8}
\input{data/predict/b/9}
\input{data/predict/b/10}
\input{data/predict/b/11}
\input{data/predict/b/12}
\input{data/predict/b/13}
\input{data/predict/b/14}

\input{data/predict/th/6}
\input{data/predict/th/7}
\input{data/predict/th/8}
\input{data/predict/th/9}
\input{data/predict/th/10}
\input{data/predict/th/11}
\input{data/predict/th/12}

\input{data/predict/t/8}
\input{data/predict/t/9}
\input{data/predict/t/10}
\input{data/predict/t/11}

\input{data/eig/b/7}
\input{data/eig/b/8}
\input{data/eig/b/9}
\input{data/eig/b/10}
\input{data/eig/b/11}
\input{data/eig/b/12}
\input{data/eig/b/13}
\input{data/eig/b/14}

\input{data/eig/th/9}
\input{data/eig/th/10}
\input{data/eig/th/11}
\input{data/eig/th/12}
\input{data/eig/th/13}
\input{data/eig/th/14}
\input{data/eig/th/16}

\input{data/eig/t/7}
\input{data/eig/t/8}
\input{data/eig/t/9}
\input{data/eig/t/10}
\input{data/eig/t/11}
\input{data/eig/t/12}
\input{data/eig/t/13}
\input{data/eig/t/14}

\input{data/proof/b/8}
\input{data/proof/b/9}
