\section{Введение}
\subsection{Основные определения}

\emph{Дистанционным графом} в $\mathbb R^n$ называется любой граф $G=(V,E)$,
у которого $V \subset \mathbb R^n, E=\{\{x,y\}: |x-y|=a\}$ с некоторым $a > 0$.
Такие графы являются предметом изучения комбинаторной геометрии, однако,
их можно встретить практически во всех разделах дискретной математики.
Основные характеристики дистанционных графов, которые нас будут
интересовать~--- это число независимости (максимальный размер подмножества
вершин графа, которое не содержит ребер):
\begin{displaymath}
    \alpha(G) = \max\{ |M|: M \subset V, \forall u,v \in M \{u,v\} \notin E \}
\end{displaymath}
и хроматическое число (минимальное число цветов, в которые можно покрасить
вершины графа):
\begin{displaymath}
    \chi(G) = \min\{C \in \mathbb N: \exists f: V \to \overline{1, C},
        \forall \{u,v\} \in E \quad f(u) \neq f(v)
    \}
\end{displaymath}
Точные значения этих величин известны лишь для узкого класса графов,
а их нахождение для конкретного графа является NP-полной задачей, чаще
всего удается лишь оценить сверху и снизу.

Число независимости графа и хроматическое число тесно связаны,
и следующая формула это наглядно демонстрирует:
\begin{equation}
    \chi(G) \geq \frac{|V|}{\alpha(G)}
\end{equation}
стоит также отметить, что для
\emph{случайного графа} эта оценка ассимптотичеси неулучшаема. \todo{ссылка}

\subsection{Проблема Нелсона-Хадвигера}

Проблема заключается в нахождении \emph{хроматического числа пространства},
которое определяется как хроматическое число дистанционного графа, множество
вершин которого в точности равно $\mathbb R^n$.
%\subsection{Проблема Борсука}
%\subsection{Другие приложения}
%Теория кодирования
\subsection{Рассматриваемые графы}

В это работе будут рассмотрены три вида дистанционных графов

\paragraph{$B(n,k,t)$:}
\begin{displaymath}
    V = \left\{
        \boldsymbol x=(x_1, \dots, x_n),
            \forall i x_i \in \{0,1\}, \sum x_i = k
    \right\}
\end{displaymath}
\begin{displaymath}
    E = \left\{
        \{u,v\} : (u,v)\footnote{
            Здесь и далее под записью $(u,v)$
            подразумевается скалярное произведение.} = t
        \right\}
\end{displaymath}
Вершины графа $B$ есть ${0,1}$-вектора с ровно $k$ единицами, а
ребра проводятся, если скалярное произведение равно $t$.

\paragraph{$T(n,k_{-1},k_1,k_0,t), k_{-1} + k_1 + k_0 = n$:}
\begin{displaymath}
    V = \left\{
        \boldsymbol x=(x_1, \dots, x_n):
            \forall i x_i \in \{-1,0,1\},
            \forall v \in {-1,0,1}, |\{i: x_i = v\}| = k_v
    \right\}
\end{displaymath}
\begin{displaymath}
    E = \left\{ \{u,v\} : (u,v) = t \right\}
\end{displaymath}
Вершины графа $T$ есть ${-1,0,1}$-вектора с фиксированным количеством ''1'' и
''-1'', а ребра проводятся, если скалярное произведение равно $t$. Стоит
отметить, что параметр $k_0$ избыточен и вводится для удобства.

\paragraph{$T_H(n,t), n=2k$:}
\begin{displaymath}
    V = \left\{
        \boldsymbol x=(x_1, \dots, x_n):
            \forall i x_i \in \{-1,0,1\},
            |\{i: x_i = 0\}| = k
    \right\}
\end{displaymath}
\begin{displaymath}
    E = \left\{ \{u,v\} : (u,v) = t \right\}
\end{displaymath}
Вершины графа $T_H$ есть ${-1,0,1}$-вектора, но в отличие от графа $T$
здесь фиксировано только количество ненулевых элементов --- $k$.
Ребра также проводятся, если скалярное произведение равно $t$.

Каждый из графов в том или ином виде был использован для получения результатов
в проблеме Эрдеша-Хадвигера, как для ассимпотических оценок, так и
для оценок в малых размерностях.
В обзорном разделе будут приведены эти результаты а также методы их получения.

\subsection{Постановка задачи}

Основная задача данной работы~--- улучшение оценок в проблеме
Нелсона-Хадвигера, а также применение новых методов для
вычисления числа независимости графа.

Целью данной работы является рассмотрение таких графов в малых размерностях,
оценка их хроматического числа и числа независимости, а в некоторых случаях и
получение точных значеный этих величин. В качестве основных инструментов
использованы вычислительные методы и перебор.
