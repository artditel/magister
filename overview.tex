\section{Обзор}
\subsection{Оценки $\Rn$}
Оценка хроматического числа пространства продвигается в двух направлениях:
ассимптотическая оценка и малые размерности.
а
Лучший ассимптотический результат на данный момент:
\begin{equation}
    (1.239\dots o(1))^n \leq \Rn \leq (3+ o(1))^n
\end{equation}
\todo{ссылка}
Лучшие нижние оценки в малых размерностях представленны в таблице
\todo{ссылка}.

\begin{table}[h]\centering
\caption{Нижняя оценка $\Rn$ в малых размерностях}
\begin{tabular}{|c|c|c|c|c|c|c|c|c|c|c|c|c|}
\hline
n &           1 & 2 & 3 & 4 & 5 & 6 &   7 & 8 & 9 & 10   & 11 & 12 \\ \hline
$\chi \geq$ & 2 & 4 & 6 & 7 & 9 & 11 & 15 & 16 & 21 & 23 & 23 & 24 \\ \hline
\end{tabular}
\end{table}


%\subsection{Статья Райгородского о хроматическом числе пространства}
%Линейно-алгебраический метод
%Всякие конструкции
\subsection{Хроматическое число $B(10,5,3)$}
В работе \todo{ссылка} были существенно улучшены оценки хроматического числа
пространства для $n=9,10$ с помощью графа $B(10,5,3)$. Было найдено точное
значение числа независимости с помощью своеобразного перебора всех вариантов.
Так как количество вершин невелико, такое удалось сделать в рамках одной
статьи, но для больших размерностей такой метод применить
невозможно. Оценка хроматического числа была улучшена с помощью понижения
размерности и других техник. \todo{каких техник}
%\subsection{Мои собственные результаты $B(n,5,2)$, $B(n,5,3)$}
%Теоремы, гипотезы
\subsection{Cпектр графа}
Собственными значениями или \emph{спектром} графа $G$ назывют собственные
значения его матрицы смежности
$\lambda_1=r \geq \lambda_2 \geq \dots \geq \lambda_n$, 
а максимальное собственное значение $r$~--- \emph{индексом графа}.
В этом параграфе приведены основные теоремы,
связывающие спектр графа и значения $\alpha(G), \chi(G)$.

\begin{theorem}
    \todo{ссылка Цветкович с.93} Число независимости графа удовлетворяет
    неравенству:
    \begin{equation}
        \alpha(G) \leq p_0 + \min(p_-, p_+),
    \end{equation}
    где $p_0,p_-,p_+$~--- количества собственных значений графа $G$
    равных нулю, меньших нуля и больших нуля соответственно. Существуют
    графы, для которых в \todo{ссылка} имеет место равенство.
\end{theorem}

\begin{theorem}
    \todo{Ссылка Цветкович с.94}
    \begin{equation}
        \chi(G) \leq r + 1.
    \end{equation} Равенство имеет место тогда и только тогда, когда $G$~---
    полный граф или протсой цикл нечетной длины.
\end{theorem}

\begin{theorem}
    \todo{Ссылка Цветкович с.95} Пусть $r (r != 0)$ и $q$~--- наибольшие и
    наименьшие собственные значения соостветственно, тогда
    \begin{equation}
        \chi(G) \geq \frac{r}{-q} + 1.
    \end{equation}
\end{theorem}

\begin{theorem}
    \todo{Ссылка Цветкович с.95} Если $G$~--- граф с $n$ вершинами и индексом
    $r$, то
    \begin{equation}
        \chi(G) \geq \frac{n}{n-r}.
    \end{equation}
\end{theorem}

К сожалению, приведенные теоремы не дают приемлемых резульатов на 
рассматриваемых в работе графах. Следуюшие теоремы применимы только к 
регулярным графам.

\begin{theorem}
    \todo{Ссылка spectra 39} Если $G$~--- регулярный граф с $n$ вершинами и
    и степенью $k$, а $\lambda_n$~--- минимальное собственное значение, то
    \begin{equation}
        \alpha(G) \leq n\frac{-\lambda_n}{k-\lambda_n},
    \end{equation}
    и если на независимом множестве $W$ достигается равенство, то любая вершина
    не из $W$ смежна с $-\lambda_n$ вершинами в $W$.
\end{theorem}
Эта теорема имеет обобщение в случае нерегулярных графов
\begin{theorem}
    \todo{Ссылка spectra 39} Если $\sigma$~--- минимальная степень в графе $G$,
    то
    \begin{equation}
        \alpha(G) \leq
            n\frac{-\lambda_1\lambda_n}{\sigma^2-\lambda_1\lambda_n}.
    \end{equation}
\end{theorem}

Функция Ловаса $\vartheta$ позволяет найти более
точную верхнуюю оценку числа независимости: пусть $A$~--- действительная
симметричная матрица порядка $n$ и $A_{ij} = 0 \Leftrightarrow i=j$ или
$(i,j) \notin E_G$, тогда
\begin{equation}
    \vartheta(G) = \min_A \lambda_{\max}(A).
\end{equation}
Функция Ловаса является супремумом емкости Шеннона графа $G$, а также
позволяет оценить число независимости графа:
\begin{theorem}
    \todo{Ссылка spectra 44, сэндвич}
    \begin{equation}
        \alpha(G) \leq \vartheta(G) \leq \chi(\overline G).
    \end{equation}
\end{theorem}

Преобразовав выражения \todo{ссыкла} получаем
\begin{equation}
    \alpha(G) \leq \lambda_{\max}(A),
\end{equation} где $A$~--- матрица, удовлетворяющая выше описанным требованиям.
Теорема \todo{ссылка на пред.теорема} может быть получена данного неравенства,
если в матрице $A$ все ненулевые элементы равны $a=1-\frac{n}{d-\lambda_n}$,
где $d$~--- степень графа, причем указанное $a$ является оптимальным.
В некотором смысле, приведенное выражение является обобщением и позволяет
улучшить оценку из теоремы \todo{ссылка}.

%\subsection{Статья Райгородского о собственном числе}
%Применение предыдущих теорем, их перебор
%\subsection{Статья Райгородского и Харламовой о $\GT$}
%\subsection{Метод Ланцоша}
%Псевдокод, пример работы, основные проблемы
