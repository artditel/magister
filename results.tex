\section{Результаты}

Основная цель~--- оценка числа независимости графа сверху с последующим
получением оценки хроматического числа пространства. Для это использовалось
три подхода:
\begin{enumerate}
    \item Перебор независимых множеств в отсечениями.
    \item Жадный метод.
    \item Оценка через минимальное собственное значение.
\end{enumerate}

\subsection{Перебор независимых множеств}

Так как время работы полного перебора экспоненциально растет с ростом $|V|$,
то такой подход можно использовать только на графах с $|V| \le 1000$.
\todo{В приложении Приложение, ссылка приведен псевдокод перебора множеств
независимости.}
В ходе перебора каждая вершина добавляется в независимое множество (если между
ней и другими вершинами нет ребер), после этого производится сравнение текущего
множества c лучшим найденным на тот момент. Если количество вершин, которые еще
можно добавить в текущее множество (несмотря на ребра между этими вершинами) в
сумме с его размером не превосходят размера лучшего множества, то перебор
останавливается.

С точки зрения программирования есть общая оптимизация: при
добавлении вершины $v$ в текущее множество сразу же помечаются только те
вершины из нерасмотренных, которые соединены ребром с $v$, а при удалении ее
из множества снимаем соответствующие отметки.

Нужно заметить, что данный метод применим на графе с большим количеством
вершин, но лишь при условии дополнительной информации.
Например, если для любой пары
вершин существует автоморфизм, переводящий одну в другую, то, зафиксировав
порядок вершин, можно полагать, что первая вершина всегда входит в множество
независимости.

Более сильные утверждения также могут сократить перебор. В частности, если
известно, что в множестве независимости всегда есть вершины $v, u: (v,u) = s$,
то взяв любые две, удовлетворящие данному условию и перенумеровав вершины,
получаем граф, в котором множество независимости всегда содержит первые две
вершины.

Если известно, что определенные вершины $v_1, \dots v_s$ входят в множество
независимости еще до начала перебора (например, в случаях, описанных выше),
то необходимо удалить их из графа $G$ вместе со всеми смежными вершинами. Таким
образом получается граф $G'$ и $\alpha(G) = \alpha(G') + s$, причем,
как правило, число удаляемых вершин много больше $s$, т.е. время перебора
сокращается на порядки.

Общая схема поиска числа независимости выглядит следущим образом ---
ищется не просто максимальное
множество независимости, а проверяется, есть ли в данном графе множество
независимости размера $a$. Таким образом, уже на начальном этапе считаетяс
что множества меньшего размера не рассматриваются, поэтому их можно отсечь
описанным выше способом.
Так как перебор можно завершить как только найдено
множество нужного размера, программа работает быстро для малых $a$. Понятно,
что для доказательства того, что $\alpha(G)=a$ необходимо проверить, что
не существует множества размера $a+1$, и перебор будет работать максимально
долго в таком случае. Однако, время работы перебора с $a=\alpha(G)$ достаточно
быстро (это связано с большим количеством максимальных множеств независимости).
Следовательно, чем дольше работает перебор, тем выше вероятность, что
он ничего не найдет, и даже остановив перебор через длительное
время, получится значение близкое к искомому. Такие
рассуждения почти всегда верны, если перебор для $a$ работает минимум на
порядок дольше, чем для $a-1$, исключение составляется случай больших графов
при $a \ll \alpha$.

Чтобы посчитать число независимости графа для больших разменостей необходимо
уменьшить количество вершин описанным выше способом. Для некоторых регулярных
графов, например $\GB$ можно считать, что любая вершина может входить в
независимое множество. Но что делать, если не
удается получить утверждение относительно некоторого множества вершин?
Для каждой размености можно попытаться найти такое множество
с помощью того же самого перебора,
который используется для поиска числа независимости. Самым простым примером
является множество из двух вершин $\{u,v\}: (u,v)=s$. В граф добавляются ребра
между вершинами, скалярное произведение которых равно $s$. Полученный таким
образом граф имеет то же количество вершин, но потенциально меньшее число
независимости, а значит и время работы перебора в таком случае сократится.
Теперь можно модифицировать наш алгоритм перебора, добавив в него шаг, на
котором мы перебираем все возможные $s$ и, найдя подходящее $s$ сможем
существенно ускорить основной перебор, добавив вершины $\{u,v\}: (u,v)=s$
в независимое множество и удалив их из графа.

Также парадоксален тот факт, что для графов с меньшим числом независимости
доказательство занимает намного меньше времени. Это объясняется меньшим
ветвлением перебора. То есть нахождение числа независимости полным перебором
имеет смысл делать только для перспективных графов и с точки зрения
производительности.

\todo{Сделать описанное улучшение для всех графов и сделать полный перебор
быстрее}

\subsection{Жадный метод}

Так как полный перебор --- затратная операция как с точки зрения процессорного
времени, так и человеческого, необходимо заранее понять, для каких графов стоит
делать его. Для этого было проведено дополнительное исследование, которое
позволило понять поведение хроматического числа для малых $n$.

Суть исследования заключалась в том, что вершины графа перенумеровывались
случайным образом и независимое множество набиралось жадно, т.е.
каждая вершина, начиная с первой, добавлялась в множество, если между ней и
текущим множеством не было ребер. Операция повторяется много раз.
Такой алгоритм дает оценку снизу на число независимости.
Если предположить, что это и есть в точности число
независимости, то получается оценка снизу на хроматическое число.
Возникают сомнения по поводу близости такой оценки к реальной, однако, как
будет видно из сравнения этих оценок с результатами перебора они достаточно
точны для графов с небольшим числом вершин.
В приложении \todo{ссылка} приведены таблицы с результатами исследования
для всех рассматриваемых графов. На их построение было потрачено несколько
месяцев процессорного времени.

\subsection{Метод собственных значений}

Метод заключается в использовании теоремы \todo{ссылка}, которая позволяет
оценить число независимости графа с помощью вычисления минимального
собственного значения матрицы смежности графа.

В отличии от перебора, сложность этого метода упирается в сложность
нахождения минимального собственного значения матрицы смежности графа.
Заметим, что такая матрица состоит из нулей и единиц, а также является
симметричной. В качестве алгоритма нахождения собственных значений была
выбрана модификация метода Ланцоша, реализованная в библиотеке ARPACK.
Сложность алгоритма - $\mathcal O(r|V|^2)$,
где $r$ --- это количество итераций алгоритма, а $|V|^2$ --- сложность
умножения мартицы смежности графа на слолбец, причем $r \ll |V|$ и зависит от
точности, с которой ищется собственное значения. Алгоритм также учитывает тот
факт, что требуется лишь одно значение, а не все.

Таким образом, быстродействие метода полностью зависит от скорости умножения
матрицы на столбец.
К сожалению, количество вершин в графе растест экспоненциально с
ростом $n$ --- размерностью пространства. В перврую очередь, алгоритм упирается
в ограничение памяти, в которой нужно хранить матрицу смежности, но этого можно
не делать, вычисля на лету и увеличив время работы примерно в $n$ раз
(столько требуется, чтобы посчитать скалярное произведение двух векторов).
Сама операция умножения
матрицы на столбец отлично параллелится --- можно независимо умножать
столбец на разные строчки вплоть до переноса шага уножения на MapReduce
кластер.  Но ни смотря на это, для графов в размерности $n \geq 18$
необходимое процессорное время могут предоставить
лишь суперкомпьютеры будущего.

Все программы написаны и на языке C++ и на Python, поэтому появилась
возможность оценить перспективность использования языка Python для переборных
задач. Результат печален: при работе в одном потоке Python работает медленнее
в 5--10 раз, а так как Python полноценно поддерживает лишь многопроцессорное
программирование, то размер потребляемой памяти увеличивается примерно в
в количество потоков (использовались 24 потока).

\subsection{$\GB$}

По приведенным результатам жадного метода можно сделать вывод: оптимальные
значения сконцентрированы на диагонали $k=t+2$, независимо от $n$, а максимум
достигается при $k=n/2$. Причем при движении по диагонали от максимума
оценка уменьшается не так быстро. Исходя из этих соображений, а также следующей
теоремы был выбран класс графов $B(n,5,3)$.
\begin{theorem}
    $\forall n \geq 10, \forall W$~--- максимальное множество независимости
    $B(n,5,3)$ $\exists u,v \in W: (u,v)=1$
\end{theorem}
\todo{Доказательство лилнейно-алгебраическим методом}

То есть, в графе $B(n,5,3)$ можно выбрать такие две вернишы
$u,v: (u,v)=1$, добавить их в множество независимости и рассматривать
граф $B(n,5,3) /\{u,v\}$. Была написана программа, которая
перебирала множества независимости описанным выше способом. Время проверки
$\alpha(B(11,5,3))=18$ --- 1.5 секунды, $\alpha(B(11,5,3))=25$ ---
несколько часов. Для $n=13$ удалось проверить лишь $\alpha(B(13,5,3))\geq33$.

Тем не менее, благодаря полученным значениям удалось существенно улучшить
нижнюю оценкую $\Rn$ для $n=10,11,12$\todo{с помощью
теорем простая оценка хроматического числа, числа пространства,
понижение размерности}:
\begin{theorem}
    \begin{eqnarray}
        \chi(\mathbb R^{10}) &\geq& \chi(B(11,5,3)) \geq
            \lceil \frac{|V|}{\alpha(B(11,5,3))} \rceil =
            \lceil \frac{C_{11}^5}{18} \rceil = \lceil25.(6)\rceil = 26 \\
        \chi(\mathbb R^{11}) &\geq& \chi(B(12,5,3)) \geq
            \lceil \frac{|V|}{\alpha(B(12,5,3))} \rceil =
            \lceil \frac{C_{12}^5}{25} \rceil = \lceil31.68\rceil = 32 \\
        \chi(\mathbb R^{12}) &\geq& \chi(B(12,5,3)) \geq = 32
    \end{eqnarray}
\end{theorem}
Таким образом, для $n=10$ оценка улучшена на 3, для 11 на 9 и для 12 на 8.


Сравнивая результаты жадного метода и метода собственных значений можно
видеть, что оценки достаточно близки и что оптимальные значения также находятся
на диагонали $k=t+2$ с максимумом при $k=n/2$.

\subsection{$\GTH$}

Перебор множеств независимости для такого класса графов не дал хороших
результатов, так как не было найдено способов сократить перебор (например,
сократив количество вершин). Однако, судя по по жадному методу максимальные
значения также расположены на диагонали $k=t+2$ и рядом с $k=n/2$.
Графы $G(n,n/2,0)$ были подробно рассмотрены в Райгородского\todo{ссылка},
где было
проведено сравнение метода собственных значений с линейно-алгебраическим
методом. В статье были приведены точные результаты, полученные с помощью
метода собственных значений для $n \leq 10$ и приблизительные для $n = 12,14$
(вычисления проводились на кластере, использовался тот же метод Ланцоша).
В рамках этой работы были получены точные результаты для $n \leq 16$.
Наибольшие сложности возникли при максимальном $n$ --- необходимо было найти
собственное значение графа с 1 647 360 вершинами,
в то время как для $n=14$ было лишь 219 648 вершин.
Матрица смежности заняла 150Gb, а значение вычислялось
2 недели и почти год процесорного времени, методу Ланцоша понадобилось 47
итераций, чтобы вычислить минимальное собственно значение с точностью до двух
знаков. Заметим замечательный факт, которому не нашлось ни объяснения, ни
применения --- минимальные собстенные значения для всех рассмотренных $n$
оказались целочисленными. Оценив с помощью полученных результатов хроматическое
число, получаем следующую таблицу, дополненную результатами с нечетным $n$:
\begin{table}[h]
    \centering
    \caption{Оценка $\chi(G(n,\lceil n/2\rceil,0))$ с помощью $\lambda_{min}$}
\begin{tabular}{|c|c|c|c|c|c|c|c|c|}
\hline
$n$ &           8  & 9   & 10 & 11   & 12  & 13 & 14 & 16 \\ \hline
$\chi \geq$   & 7  & 9   &  9 & 11   & 11  & 13 & 13 & 15 \\ \hline
$\alpha \leq$ & 78 & 227 &465 & 1428 &2563 & 8903 & 16471 & 106744 \\ \hline
\end{tabular}
\end{table}

Из этой таблицы можно сформулировать гипотезу
\begin{hyposesis}
    \begin{displaymath}
        \chi(G(n,\lceil n/2\rceil,0)) \geq \left\{ \begin{array}{ll}
            n-1 & \textrm{если $n$ четно}\\
            n & \textrm{если $n$ нечетно}
        \end{array} \right.
    \end{displaymath}
\end{hyposesis}

Глядя на гипотезу, можно строить предположения о том, что для других, более
эффективных с точки зрения хроматического числа графов удастся получить
аналогичные результаты.

\subsection{$\GT$}

Подобные графы очень сложно оценить с помощью перебора по причине большого
количества вершин --- уже в размерности 8 вершин получается больше, чем у
$G(11, 5 ,3)$. Единственный метод, результаты которого можно
проанализировать~--- оценка через собственные значения.
Судя по таблицам для $n=12,13$, максимальные значения получаются при
$k_{-1} \approx k_1 \approx t$ и $k_{-1} + k_1 \approx n/2$.
Причем оценка для графа $T$ намного лучше аналогичных для графов $G$ и $T_H$,
и почти такая же как и оценка через точное значение $\alpha(\GB)$.
Такой тип графов являтеся наиболее перспективным в будущем.
