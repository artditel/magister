\documentclass[svgnames,table]{beamer}
\usepackage[utf8]{inputenc}
\usepackage[english,russian]{babel}
\usepackage{amsfonts}
\usepackage{amsmath}
%\usepackage[table]{xcolor}
\usetheme{Darmstadt}
\usecolortheme{seahorse}
\usefonttheme{serif}


\title % (optional, only for long titles)
{Хроматические числа графов расстояний}
\subtitle{Дипломная работа}
\institute[МФТИ] % (optional)
{
Московский физико-технический институт
}
\date[2013] % (optional)
{2013}
\subject{Дискретная математика}

\newcommand\GTH{T_H(n,k,t)}
\newcommand\GB{B(n,k,t)}
\newcommand\GT{T(n,k_{-1},k_1,k_0,t)}
\newcommand\Rn{\chi(\mathbb R^n)}
\setbeamerfont{title}{family=\rm}

\newtheorem{proposition}{Теорема}
\begin{document}
\frame{\titlepage}

\begin{frame}
    \frametitle{Определения}
    \begin{itemize}
    \item\emph{Дистанционный граф} в $\mathbb R^n$ --- любой граф,
    у которого $V \in \mathbb R^n, E=\{\{x,y\}: |x-y|=a\}$ с некоторым $a > 0$.

    \item\emph{Хроматическое число} графа $\chi(G)$---
    минимальное число цветов для покраски вершин,
    при которой концы ребер имеют разные цвета.

    \item\emph{Хроматическое число пространства} $\mathbb R^n$ ---
    это хроматическое число дистанционного графа с $V=\mathbb R^n$

    \item\emph{Число независимости графа} $\alpha(G)$---
    максимальный размер подмножества вершин, не содержащего ребер
    \end{itemize}
\end{frame}

\begin{frame}
\frametitle{Постановка задачи}
\framesubtitle{Проблема Нелсона—Эрдеша—Хадвигера}
    \begin{itemize}
    \item Нужно найти $\Rn$, или хотя бы оценить.
    \item На данный момент
        $(1.239\dots o(1))^n \leq \Rn \leq (3+ o(1))^n$
    \item Для любого дистанционного графа $G$ из $\mathbb R^n$ верно:
        $$\Rn \geq \chi(G) \geq \frac{|V|}{\alpha(G)}$$
    \item Будем искать нижнюю оценку в малых размерностях.
    \end{itemize}
\end{frame}

\begin{frame}
    \frametitle{Известные результаты}
    \begin{table}\centering
        \caption{Нижняя оценка $\Rn$ в малых размерностях}
    \begin{tabular}{|c|c|c|c|c|c|c|c|c|c|c|c|c|}
        \hline
        n &         1 & 2 & 3 & 4 & 5 & 6 &
        7 & 8 & 9 & 10 & 11 & 12 \\ \hline
        $\chi \geq$ & 2 & 4 & 6 & 7 & 9 & 11 &
         15 & 16 & 21 & 23 & 23 & 24 \\ \hline
    \end{tabular}
\end{table}
\end{frame}

\begin{frame}
    \frametitle{Рассматриваемые графы}
    \begin{itemize}
        \item $\GB$: вершины --- это $\{0,1\}$-вектора с $k$ единицами,
            $\{u,v\}$--- ребро, если $(u,v)=t$.
        \item $\GTH$: вершины --- это $\{-1,0,1\}$-вектора с $k$ ненулевыми
            элементами, $\{u,v\}$--- ребро, если $(u,v)=t$. Считаем, что 1
            идет первой.
        \item $\GT$: похож на $\GTH$, но количество ''$-1$'' фиксировано.
    \end{itemize}
\end{frame}

\begin{frame}
    \frametitle{Применяемые методы}
    \framesubtitle{Жадная прикидка}
        Будем брать вершины в случайном порядке и добавлять в множество
            независимости. Повторять много раз.
            \input{data/predict/b/10}
\end{frame}

\begin{frame}
    \frametitle{Применяемые методы}
    \framesubtitle{Полный перебор}
        \begin{itemize}
            \item Перебор с отсечениями.
            \item Докажем, что множество, в котором нет пары таких вершин,
                что $(u,v)=s$ --- не максимальное.
            \item Возмьем две такие веришны и добавим их в множество,
                удалив из графа всех соседей.
            \item Теперь в графе намного меньше вершин.
        \end{itemize}
\end{frame}

\begin{frame}
    \frametitle{Полный перебор}
    \framesubtitle{Результаты}
        \begin{itemize}
            \item Граф $B(n,5,3), n=11,12$, время проверки --- несколько часов.
            \item Так как граф на гиперсфере, то мы можем понизить его размерность.
        \end{itemize}
    \begin{eqnarray}
        \chi(\mathbb R^{10}) &\geq& \chi(B(11,5,3)) \geq  26 \textrm{ (было 23)},\\\
        \chi(\mathbb R^{11}) &\geq& \chi(B(12,5,3)) \geq  32 \textrm{ (было 23)},\\
        \chi(\mathbb R^{12}) &\geq& \chi(B(12,5,3)) \geq  32 \textrm{ (было 24)}.
    \end{eqnarray}
\end{frame}

\begin{frame}
    \frametitle{Применяемые методы}
    \framesubtitle{Собственные значения}
    \begin{proposition}
        Если $G$~--- регулярный граф с $n$ вершинами
        и степенью $k$, а $\lambda_n$~--- минимальное собственное значение
        его матрицы смежности, то
        \begin{equation}
            \alpha(G) \leq n\frac{-\lambda_n}{k-\lambda_n}.
        \end{equation}
    \end{proposition}
    Применим метод Ланцоша для нахождения собственного значения.
    Сложность алгоритма --- $\mathcal O(r|V|^2)$, все упирается в умножение
    матрицы на столбец.
\end{frame}

\begin{frame}
    \frametitle{Собственные значения}
    \framesubtitle{Результаты}
    \begin{itemize}
        \item В другой статье был рассмотрен граф $T_H(n,n/2,0)$, и для него
            применен подобный метод. Точные значения получены для $ n \leq 10$
            и приблизительные для $n=12,14$
        \item Для $n=16$ граф с 1 647 360 вершинами,
           для $n=14$ --- лишь 219 648 вершин.  Матрица смежности заняла 150Gb,
           значение вычислялось 2 недели, год процесорного времени. Методу Ланцоша понадобилось 47 итераций.
    \end{itemize}
    \begin{table}[h]
        \centering
        \caption{Оценка $\chi(G(n,\lceil n/2\rceil,0))$ с помощью $\lambda_{min}$}
    \begin{tabular}{|c|c|c|c|c|c|c|c|c|}
    \hline
    $n$ &           8  & 9   & 10 & 11   & 12  & 13 & 14 & 16 \\ \hline
    $\chi \geq$   & 7  & 9   &  9 & 11   & 11  & 13 & 13 & 15 \\ \hline
    $\alpha \leq$ & 78 & 227 &465 & 1428 &2563 & 8903 & 16471 & 106744 \\ \hline
    \end{tabular}
    \end{table}

\end{frame}

\begin{frame}
    \frametitle{Конец}
    \begin{center}
    Спасибо за внимание.
    \end{center}

\end{frame}

\end{document}
